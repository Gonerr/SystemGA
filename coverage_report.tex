\documentclass{article}
\usepackage[utf8]{inputenc}
\usepackage[russian]{babel}
\usepackage{booktabs}
\usepackage{float}
\usepackage{listings}
\usepackage{enumitem}
\usepackage{graphicx}

\title{Отчет о покрытии кода тестами}
\author{Анализ игровой индустрии}
\date{\today}

\begin{document}

\maketitle

\section{Общая информация}
В данном отчете представлены результаты тестирования основных компонентов системы анализа игровой индустрии. Система включает в себя несколько ключевых модулей, каждый из которых отвечает за определенный аспект анализа.

\section{Покрытие кода тестами}
В таблице \ref{tab:coverage} представлены данные о покрытии кода тестами для каждого модуля системы.

\begin{table}[H]
\centering
\caption{Покрытие кода тестами по модулям}
\label{tab:coverage}
\begin{tabular}{lrrr}
\toprule
Модуль & Строк кода & Пропущено & Покрытие \\
\midrule
BaseAnalyzer & 159 & 95 & 40\% \\
DataRepository & 70 & 25 & 64\% \\
FeatureExtractor & 160 & 15 & 91\% \\
GameAnalysisPipeline & 174 & 45 & 74\% \\
GameClusterer & 229 & 45 & 80\% \\
GameCompetitivenessAnalyzer & 121 & 20 & 83\% \\
GameRecommendationAnalyzer & 141 & 35 & 75\% \\
GameSimilarityAnalyzer & 98 & 25 & 74\% \\
Decorators & 18 & 1 & 94\% \\
\midrule
\textbf{Итого} & \textbf{1310} & \textbf{306} & \textbf{77\%} \\
\bottomrule
\end{tabular}
\end{table}

\begin{figure}[H]
    \centering
    \includegraphics[width=0.9\linewidth]{my_folder/images/Тест_1.jpg}
    \caption{Результаты выполнения модульных тестов: все 12 тестовых функций успешно пройдены (12 passed, 0 failed). Тесты охватывают все ключевые аспекты работы системы, включая анализ рекомендаций, ценовое позиционирование, уникальные особенности и обработку граничных случаев.}
    \label{fig:tests}
\end{figure}

\section{Анализ результатов}
Основные аналитические методы системы демонстрируют высокий уровень покрытия тестами:

\begin{itemize}
    \item Модуль кластеризации игр (GameClusterer) - 80\% покрытия
    \item Анализ конкурентной способности (GameCompetitivenessAnalyzer) - 83\% покрытия
    \item Поиск похожих игр (GameSimilarityAnalyzer) - 74\% покрытия
\end{itemize}

Особенно высокое покрытие тестами (91\%) демонстрирует модуль извлечения признаков (FeatureExtractor), что обеспечивает надежность базового функционала системы.

\section{Тестовые сценарии}
\begin{enumerate}[label=\arabic*.]
    \item Тестирование передачи данных между FeatureExtractor и GameClusterer:
    \begin{itemize}
        \item Входные данные: массив необработанных данных.
        \item Ожидаемый результат: корректный формат данных, передаваемых в GameClusterer.
        \begin{lstlisting}[language=Python]
def test_data_flow_between_extractor_and_clusterer(feature_extractor, clusterer):
    sample_data = [
        {"description": "An epic <b>RPG</b> with dragons!", "price": 59.99},
        {"description": "<p>Fast-paced shooter</p>", "price": 19.99}
    ]
    
    processed_data = feature_extractor.preprocess_data(sample_data)
    cluster_labels = clusterer.perform_clustering(processed_data, n_clusters=2)
    
    assert len(cluster_labels) == len(processed_data)
    assert all(isinstance(label, int) for label in cluster_labels)
        \end{lstlisting}
    \end{itemize}

    \item Тестирование передачи данных между GameClusterer и GameCompetitivenessAnalyzer:
    \begin{itemize}
        \item Входные данные: кластеризованные данные и веса для расчета CI.
        \item Ожидаемый результат: корректные индексы конкурентоспособности для каждого кластера.
        \begin{lstlisting}[language=Python]
def test_data_flow_between_clusterer_and_analyzer(clusterer, analyzer):
    sample_features = np.array([
        [0.5, 0.8],
        [0.6, 0.75],
        [0.2, 0.9]
    ])
    
    cluster_labels = clusterer.perform_clustering(sample_features, n_clusters=2)
    weights = [0.4, 0.6]
    ci_values = analyzer.calculate_competitiveness_index(sample_features, weights)
    
    assert len(ci_values) == len(sample_features)
    assert all(0 <= ci <= 1 for ci in ci_values)
        \end{lstlisting}
    \end{itemize}

    \item Тестирование обработки некорректных данных (класс FeatureExtractor):
    \begin{itemize}
        \item Входные данные:
        \begin{lstlisting}[language=Python]
sample_data = [
    {"description": "An epic <b>RPG</b> with dragons!", "price": 59.99, "metacritic": 85, "median_forever": 120},
    {"description": "<p>Fast-paced action shooter.</p>", "price": 19.99, "metacritic": 70, "median_forever": 50}
]
        \end{lstlisting}
        \item Ожидаемый результат после предобработки:
        \begin{lstlisting}[language=Python]
expected_output = [
    {"description": "an epic rpg with dragons", "price": 0.75, "metacritic": 0.85, "median_forever": 0.6},
    {"description": "fast-paced action shooter", "price": 0.25, "metacritic": 0.7, "median_forever": 0.25}
]
        \end{lstlisting}
        \item Тестовый сценарий:
        \begin{lstlisting}[language=Python]
def test_data_preprocessing():
    processed_data = preprocess_data(sample_data)
    assert processed_data == expected_output
        \end{lstlisting}
    \end{itemize}

    \item Тестирование алгоритмов кластеризации (класс GameClusterer):
    \begin{itemize}
        \item Входные данные:
        \begin{lstlisting}[language=Python]
sample_features = [
    [0.2, 0.8],
    [0.25, 0.75],
    [0.7, 0.3],
    [0.72, 0.28],
    [0.8, 0.2]
]
        \end{lstlisting}
        \item Ожидаемый результат кластеризации (для 2 кластеров):
        \begin{lstlisting}[language=Python]
Cluster 1: {[0.2, 0.8], [0.25, 0.75]}
Cluster 2: {[0.7, 0.3], [0.72, 0.28], [0.8, 0.2]}
        \end{lstlisting}
        \item Тестовый сценарий:
        \begin{lstlisting}[language=Python]
def test_clustering_algorithm():
    clusters = perform_clustering(sample_features, n_clusters=2)
    assert len(clusters[0]) == 2
    assert len(clusters[1]) == 3
        \end{lstlisting}
    \end{itemize}

    \item Тестирование расчета индекса конкурентоспособности:
    \begin{itemize}
        \item Входные данные:
        \begin{lstlisting}[language=Python]
sample_games = [
    {
        "name": "Game1",
        "price": 0.5,
        "metacritic": 0.8,
        "median_forever": 0.7,
        "features": [0.5, 0.8, 0.7]
    },
    {
        "name": "Game2",
        "price": 0.6,
        "metacritic": 0.9,
        "median_forever": 0.75,
        "features": [0.6, 0.9, 0.75]
    }
]
        \end{lstlisting}
        \item Ожидаемый индекс конкурентоспособности (при весах 0.4, 0.3, 0.3):
        \begin{itemize}
            \item Game 1: $CI=0.5\cdot0.4+0.8\cdot0.3+0.7\cdot0.3=0.65$
            \item Game 2: $CI=0.6\cdot0.4+0.9\cdot0.3+0.75\cdot0.3=0.73$
        \end{itemize}
        \item Тестовый сценарий:
        \begin{lstlisting}[language=Python]
def test_competitiveness_index():
    weights = [0.4, 0.3, 0.3]
    ci_values = calculate_competitiveness_index(sample_games, weights)
    assert round(ci_values[0], 2) == 0.65
    assert round(ci_values[1], 2) == 0.73
        \end{lstlisting}
    \end{itemize}

    \item Тестирование уникальных особенностей игр:
    \begin{itemize}
        \item Входные данные:
        \begin{lstlisting}[language=Python]
sample_user_game = {
    "data": {
        "categories": [
            {"description": "Single-player"},
            {"description": "Multi-player"}
        ],
        "genres": ["RPG", "Action", "Adventure"],
        "name": "Test Game",
        "detailed_description": "An epic RPG with dragons and unique combat system!",
        "metacritic": {"score": 85},
        "median_forever": 120
    }
}
        \end{lstlisting}
        \item Тестовый сценарий:
        \begin{lstlisting}[language=Python]
def test_unique_features_analysis():
    comparison = analyzer.compare_with_most_similar_and_dissimilar(
        sample_user_game, sample_similar_games)
    
    assert isinstance(comparison, dict)
    assert 'size' in comparison
    assert 'features' in comparison
    assert comparison['size'] > 0
    
    features = comparison['features']
    assert isinstance(features, dict)
    for game_id, feature_data in features.items():
        assert 'description_similarity' in feature_data
        assert 'genre_similarity' in feature_data
        assert 'category_similarity' in feature_data
        assert 'overall_similarity' in feature_data
        \end{lstlisting}
    \end{itemize}

    \item Тестирование конвейера анализа игр:
    \begin{itemize}
        \item Входные данные:
        \begin{lstlisting}[language=Python]
sample_game = {
    "name": "Test Game",
    "description": "An epic RPG with dragons!",
    "price": 59.99,
    "metacritic": 85,
    "median_forever": 120,
    "genres": ["RPG", "Action", "Adventure"],
    "categories": ["Single-player", "Multi-player"],
    "tags": ["RPG", "Action", "Fantasy"]
}
        \end{lstlisting}
        \item Тестовый сценарий:
        \begin{lstlisting}[language=Python]
def test_game_analysis_pipeline():
    results = pipeline.analyze_game(sample_game)
    
    assert isinstance(results, dict)
    assert 'cluster_analysis' in results
    assert 'competitiveness_analysis' in results
    assert 'similarity_analysis' in results
    assert 'unique_features' in results
    
    # Проверка согласованности результатов
    results2 = pipeline.analyze_game(sample_game)
    assert results == results2
        \end{lstlisting}
    \end{itemize}

    \item Тестирование обработки пустых данных:
    \begin{itemize}
        \item Входные данные:
        \begin{lstlisting}[language=Python]
empty_game = {
    "name": "",
    "description": "",
    \item Тестовый сценарий:
    \begin{lstlisting}[language=Python]
def test_empty_data_handling():
    results = pipeline.analyze_game(empty_game)
    assert isinstance(results, dict)
    assert all(key in results for key in [
        'cluster_analysis',
        'competitiveness_analysis',
        'similarity_analysis',
        'unique_features'
    ])
        \end{lstlisting}
    \end{itemize}

    \item Тестирование анализатора рекомендаций игр:
    \begin{itemize}
        \item Входные данные:
        \begin{lstlisting}[language=Python]
sample_user_game = {
    "name": "Test Game",
    "description": "An epic RPG with dragons!",
    "price": 59.99,
    "metacritic": 85,
    "median_forever": 120,
    "genres": ["RPG", "Action", "Adventure"],
    "categories": ["Single-player", "Multi-player"],
    "tags": ["RPG", "Action", "Fantasy"]
}

sample_similar_data = {
    "all_genres": ["RPG", "Action", "Strategy"],
    "all_tags": ["RPG", "Action", "Fantasy", "Open World"],
    "all_categories": ["Single-player", "Multi-player", "Co-op"],
    "unique_in_target_genres": ["Adventure"],
    "unique_in_target_tags": ["Dragons"],
    "description_analysis": {
        "similarity_score": 0.75,
        "common_phrases": ["epic adventure", "fantasy world"],
        "unique_phrases": ["dragon companions", "magic system"]
    }
}
        \end{lstlisting}
        \item Тестовый сценарий:
        \begin{lstlisting}[language=Python]
def test_recommendation_analysis():
    recommendations = analyzer.generate_recommendations(
        sample_user_game, 
        sample_similar_data, 
        sample_dissimilar_data
    )
    
    assert isinstance(recommendations, list)
    assert len(recommendations) > 0
    
    # Проверка наличия всех разделов рекомендаций
    sections = [
        "Позиционирование на рынке",
        "Уникальные особенности",
        "Ценовое позиционирование",
        "Конкурентные преимущества",
        "Рекомендации по описанию",
        "Потенциальные улучшения"
    ]
    
    for section in sections:
        assert any(section in rec for rec in recommendations)
        \end{lstlisting}
    \end{itemize}

    \item Тестирование анализа ценового позиционирования:
    \begin{itemize}
        \item Входные данные:
        \begin{lstlisting}[language=Python]
# Тест с низкой ценой
sample_user_game["price"] = 29.99
result_low = analyzer._analyze_price_positioning(
    sample_user_game, 
    sample_similar_data, 
    sample_dissimilar_data
)

# Тест с высокой ценой
sample_user_game["price"] = 89.99
result_high = analyzer._analyze_price_positioning(
    sample_user_game, 
    sample_similar_data, 
    sample_dissimilar_data
)
        \end{lstlisting}
        \item Тестовый сценарий:
        \begin{lstlisting}[language=Python]
def test_price_positioning():
    assert "ниже среднерыночной" in result_low
    assert "выше среднерыночной" in result_high
    assert isinstance(result_low, str)
    assert isinstance(result_high, str)
        \end{lstlisting}
    \end{itemize}

    \item Тестирование анализа уникальных особенностей:
    \begin{itemize}
        \item Входные данные:
        \begin{lstlisting}[language=Python]
sample_user_game["genres"] = ["RPG", "Strategy", "Simulation"]
sample_user_game["tags"] = ["UniqueTag1", "UniqueTag2"]
        \end{lstlisting}
        \item Тестовый сценарий:
        \begin{lstlisting}[language=Python]
def test_unique_features():
    combinations = analyzer._analyze_feature_combinations(
        sample_user_game, 
        sample_similar_data
    )
    advantages = analyzer._analyze_competitive_advantages(
        sample_user_game, 
        sample_similar_data, 
        sample_dissimilar_data
    )
    
    assert isinstance(combinations, list)
    assert isinstance(advantages, list)
    assert any("Уникальная комбинация жанров" in comb for comb in combinations)
    assert any("Уникальные теги" in adv for adv in advantages)
        \end{lstlisting}
    \end{itemize}

    \item Тестирование анализа описания:
    \begin{itemize}
        \item Входные данные:
        \begin{lstlisting}[language=Python]
# Тест с высокой схожестью
sample_similar_data["description_analysis"]["similarity_score"] = 0.9
result_high_sim = analyzer._analyze_description(
    sample_user_game, 
    sample_similar_data, 
    sample_dissimilar_data
)

# Тест с низкой схожестью
sample_similar_data["description_analysis"]["similarity_score"] = 0.2
result_low_sim = analyzer._analyze_description(
    sample_user_game, 
    sample_similar_data, 
    sample_dissimilar_data
)
        \end{lstlisting}
        \item Тестовый сценарий:
        \begin{lstlisting}[language=Python]
def test_description_analysis():
    assert isinstance(result_high_sim, list)
    assert isinstance(result_low_sim, list)
    assert any("очень похоже" in rec for rec in result_high_sim)
    assert any("значительно отличается" in rec for rec in result_low_sim)
        \end{lstlisting}
    \end{itemize}

    \item Тестирование обработки пустых данных:
    \begin{itemize}
        \item Входные данные:
        \begin{lstlisting}[language=Python]
empty_game = {
    "name": "",
    "description": "",
    "price": 0,
    "metacritic": 0,
    "median_forever": 0,
    "genres": [],
    "categories": [],
    "tags": []
}

empty_similar_data = {
    "all_genres": [],
    "all_tags": [],
    "all_categories": [],
    "unique_in_target_genres": [],
    "unique_in_target_tags": [],
    "unique_in_target_categories": [],
    "intersection_genres": [],
    "intersection_tags": [],
    "intersection_categories": [],
    "avg_price_diff": 0,
    "description_analysis": {
        "similarity_score": 0,
        "common_phrases": [],
        "unique_phrases": []
    }
}
        \end{lstlisting}
        \item Тестовый сценарий:
        \begin{lstlisting}[language=Python]
def test_empty_data_handling():
    recommendations = analyzer.generate_recommendations(
        empty_game, 
        empty_similar_data, 
        empty_dissimilar_data
    )
    
    assert isinstance(recommendations, list)
    assert len(recommendations) > 0
    assert "общую рекомендацию" in recommendations[-1]
        \end{lstlisting}
    \end{itemize}
\end{enumerate}

\section{Заключение}
Общее покрытие кода тестами составляет 77\%, что является хорошим показателем для системы такого масштаба. Особенно важно отметить высокое покрытие тестами ключевых аналитических модулей, что обеспечивает надежность и стабильность работы системы.

\end{document} 